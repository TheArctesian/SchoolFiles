\documentclass{article}
\usepackage[a4paper, total={6in, 10in}]{geometry}
\usepackage{hyperref}
\hypersetup{
    colorlinks=true,
    linkcolor=blue,
    filecolor=magenta,      
    urlcolor=cyan,
    pdftitle={Overleaf Example},
    pdfpagemode=FullScreen,
    }
\begin{document}

\title{ERQ Prompt: To what extent do pheromones effect human behavior}
\author{Stephen Okita}
\date{Dec 2021}
\maketitle
% my latex compiler is broken so I'm using just going to use this
Pheromones are a chemical substance produced and released into the environment by an animal affecting the behavior or physiology of other of its own species.  Although pheromones are known to play a significant role in signaling between members of the same species among animals to affect various behaviors, it is not clear that is also true in humans. At this stage no conclusive evidence has been found to show the pheromones affect human behavior.

A double blind study was done be Wedekind (1995)  nicknamed the “Sweaty T-shirt Study”  to investigate whether a man's MHC,  large group of genes responsible for the development of the immune system, effects a women's choice in when smelling a man shirt.  The researchers assembled volunteers, 49 women and 44 men selected for their variety of MHC gene types. Men were given clean shirts to wear for 2 nights before they were returned to the scientists. The men were asked to abstain from spicy foods and other activities that would affect the smell of the shirt, in addition they were given odor free deodorants and kept the shirts in plastic bags when they weren't wearing them. The women were asked to use a nose spray 14 day before the experiment to support regeneration of the nasal membrane and prevent common colds. Each woman was given a copy of Susskind's novel perfume to sensitize their smell perception. During the experiment seven the women were asked to rank the of 7 shirts from cardboard boxes through a smelling hole. 3 of the seven boxes contained shirts from men with similar MHC to the women's, 3 contend shirts from dissimilar men, the final box was an unworn shirt as a control. With each woman being alone, they were asked to score the odors from 0-10 in the criteria of intensity, pleasantness and sexiness. 

From this the researchers found the women scored male body odors with dissimilar MHC then their own as more pleasant. Additionally, the women were noted saying odors of dissimilar MHC remind them of former mates suggesting that the experiment had high ecological validity.  This study suggests that MHC could affect the human behavior of mate selection, as women were noted to prefer scents with dissimilar MHC. Thought it does show that pheromones' possibly have an effect on the human behavior of mate selection, the researchers were not able to synthesizes a specific pheromone and therefore does not conclusively show that pheromones' exist in humans. 

However as pheromones have not been proved in humans, the effects of pheromones on behavior could be subject to researcher basis (errors in the research process caused by the researcher expectation or preconceived beliefs) as seen in McClincktok (1971) study on pheromones effects on menstrual synchronicity. The study believed that convincing evidence showing the menstrual synchronicity would justify the effect pheromones have on human behavior. 

The study analyzed the menstrual synchronicity in 135 women living in a college dorm. The study used questionnaires to determine the onset dates (data in which a women's period begins) of the dormmates menstrual cycles. The study reported that onsets for dormitory friends became 2 days closer together over a 4–6 month period. Providing evidence for the effect pheromones have on the human behaviour of menstrual synchronicity. 

However, when further analysing the study, there are 3 clear errors in the research design. 

\begin{enumerate}
    \item Failure to recognize synchronicity by chance
    \item Miscalculation in menstrual onsets
    \item Exclusion from the sample of onsets of subjects in order to fit the number of onsets to the specifications of the research design
\end{enumerate}

Furthermore the study was repeated in two different experiments Trevathan et al (1993) and Yang & Schank (2006), in which no evidence was found supporting menstrual synchronicity. Thus showing the effect researcher bias has when determining the effects pheromones have on human behaviours.

In summary, the effects of pheromones on humans is currently unclear, to this date studies show a possible correlation, Wedekind(1995), others show little to no correlation; however no study has been able to synthesis a pheromone, therefore we can not provide concrete evidence the effect the hormones have on human behavior.

\end{document}