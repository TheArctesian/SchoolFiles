\documentclass{article}
\usepackage[a4paper, total={6in, 10in}]{geometry}
\usepackage{hyperref}
\usepackage{amsmath}
\usepackage{siunitx}
\usepackage{mhchem}
\newcommand\hr{\par\vspace{-.5\ht\strutbox}\noindent\hrulefill\par}
\newcommand\br{\vspace{0.5cm}}
\begin{document}
\Huge \raggedright Thermal Physics \\ \vspace{1cm}
\begin{large}
    \hr
    \LARGE \center Atoms \\
    \br
    
    \raggedright
    \large Atoms consist of electrons and protons. The electrons are negatively charged and the protons are positively charged. The size of one atom is around 0.1nm or \num{0.1e-9} meters known as one Ångström. \\
    \large A neutral atom has an equal number of protons and electrons, referred to as its atomic number, When an atom loses or gains an electron itt becomes an ion. When noting isotopes we use \ce{^{A}_{Z}X+} where Z in the proton number, A is the nucleon number and X is the chemical symbol. \\ 
    \large The number of neutrons can be found with \{N = A - Z\}. \\ 
    \large as Atoms are very small the amount of atoms can be defined with a mole or avogadro's constant \num{6.0221415e23} atoms per mole. \\
    \large atoms mass can be represented with the atomic mass unit \num{1.6605402e-27} kg. \\
    \large Molar mass, M (g/mol), is mass per mole. \[M \equiv \frac{m}{n}\] where m is the mass of an atom and n is the number of atoms in a mole. \\




    \hr
    \LARGE \center Laws \\
    \br
    \raggedright
    



    \hr
    \LARGE \center Ideal Gas Law \\
    \br
    \raggedright



    \hr
    \LARGE \center Phase Changes \\
    \br
    \raggedright

    \hr

\end{large}

\end{document}