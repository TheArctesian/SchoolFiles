\documentclass{article}
\usepackage{amsmath}
\usepackage[a4paper, total={6in, 10in}]{geometry}
\usepackage{hyperref}
\usepackage{amsmath}
\usepackage{siunitx}
\usepackage[version=4]{mhchem}
\usepackage{amssymb}
\usepackage{inputenc}

\newcommand\hr{\par\vspace{-.5\ht\strutbox}\noindent\hrulefill\par}
\newcommand\br{\vspace{0.5cm}}
\begin{document}
\Huge \raggedright Electrical Circuts \\ \vspace{1cm}
    \hr
    \LARGE \center Electric Current \\
    \br
    \begin{large}
    \raggedright 
    Voltage is the rate of change of electric potential. \\ 
    \[ V \equiv\frac{w}{q}\] measured in voltage $\frac{J}{C} = V$
    
    Current is defined as the amount of charge flowing through a cross sectional area per uint time: 
    \[ I \equiv \frac{\Delta Q}{\Delta t}\] measured in amperes:  $\frac{C}{S} = A$.
    
    Resitance is the ratio of the voltage drop to the current flow through a conductor:
    \[ R \equiv \frac{V}{I}\] measured in ohms: $\frac{V}{A} = \Omega$.
    
    Fraaday Cage, A circut with a copper wire with have equally charged haowever when connected to a battery EMF is the energy per unit charge delivered to the circit when a given charge travels 
    Current through the wire. 
    \end{large}
    \LARGE \center Drift Speed Formula \\
    \br
    \begin{large}
    \raggedright
    During a small frame find how far a charge moves so $V*\Delta T$ is the distance traveled. Then volume of the an area wher $Av*\Delta T$ is the distance traveled. Then the new charges crossed the area A and now reside in the new volume the amount of charges is $n*Av\delta t$. The total amount of charge is $q*nAv\Delta t$ so the final current would be \[I = qnAv\]
    To calculate the drift speed of a particle we use the formula:
    \[ \frac{v}{c} = \frac{v_0}{c_0} \]
    where v is the velocity in m/s and c is the speed of light in m/s.
    
    \end{large}

    


\end{document}